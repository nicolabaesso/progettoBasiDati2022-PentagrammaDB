\documentclass[12pt]{article}
\fontfamily{Arial}
\usepackage{graphicx} %necessario per l'inserimento delle immagini
%opening
\title{Progetto Basi di Dati}
\author{Baesso Nicola, Egidi Annalisa}

\begin{document} %inizia il documento vero e proprio

\maketitle %crea il titolo
\begin{center} %mette il contenuto in centro
	\includegraphics{logo.jpeg}	
\end{center}
\newpage %crea una nuova pagina (chiudendo la precedente)
\tableofcontents %crea la tabella dei contenuti
\newpage
\section{Abstract}
La Scuola di Musica Il Pentagramma è un’associazione culturale senza fini di lucro che dal 1982 ha lo scopo di promuovere, diffondere e sviluppare la conoscenza, l’apprendimento ed il perfezionamento della musica moderna e classica in tutte le sue tecniche strumentali e canore.\\
L'associazione organizza corsi, senza limiti di età, per l'apprendimento di tecniche strumentali o vocali, di registrazione, composizione e di tutto ciò che sia collegato alla musica. Inoltre promuove la creazione di gruppi musicali e organizza eventi e concerti in proprio o in accordo con locali o enti privati e locali. Inoltre, offre ai soci sconti e agevolazioni per eventi e software musicali.\\
I corsi, suddivisi tra corsi individuali, musica d'insieme, teatro e corsi propedeutici, sono tenuti da docenti diplomati al conservatorio e alcuni sono svolti anche dagli stessi dirigenti della scuola. Hanno durata di 30,45 o 60 minuti a seconda della tipologia del corso.\\
L'associazione vuole costruire una base di dati affinché possa organizzare al meglio le attività dell'associazione, oltre che mantenere uno storico delle attività passate.

\section{Analisi dei requisiti}
	\subsection{Descrizione}
	\subsection{Glossario}
	\subsection{Strutturazione}
\section{Progettazione concettuale}
	\subsection{Lista delle entità}
	\subsection{Lista delle associazioni}
	\subsection{Lista delle generalizzazioni}
	\subsection{Schema ER}

\end{document}
